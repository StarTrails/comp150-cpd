% Please do not change the document class
\documentclass{scrartcl}

% Please do not change these packages
\usepackage[hidelinks]{hyperref}
\usepackage[none]{hyphenat}
\usepackage{setspace}
 \doublespace

% You may add additional packages here
\usepackage{amsmath}

% Please include a clear, concise, and descriptive title
\title{Reviewing my first term at university}

% Please do not change the subtitle
\subtitle{COMP150 - CPD Report}

% Please put your student number in the author field
\author{1707502}

\begin{document}

\maketitle

\section{Introduction}
In this document I will be talking about my experiences in the first term at university and skills that i wish to improve on throughout the year ahead of me.
Although i currently do not have a strict career path in mind for my future within the games industry I am very interested in taking part in the creation of games in general. The skills that I will be talking about in this review are academic writing structuring, sharing and critiquing work,completing academic research effectively and further development of programming knowledge.   

\section{Presentation skills}
Having a programming job in the games industry requires allot more social ability than one would think. Becoming more confident in presenting work to my cohorts will help me become more sociable in a business environment and give me more of  an edge over my fellow programmers searching for post Graduate roles. Being able to communicate about work related tasks efficiently will help increase my productivity in a work place in the future.
I believe that our pillars presentation was lacking heavily in talking about the funding of the project itself  In order to improve my presentation skills for the following term I will be periodically reviewing material instructing about tried and tested presentation techniques. 


\section{Sharing and critiquing work}

During our group project for this term I have found that being able to communicate effectively in a team environment is very important. Especially when it comes to critiquing lack of or lower quality work but being constructive enough that it doesn't strongly offend the creator of that particular workload. I feel that issues with critiquing work correctly affected my teams Comp-150 project greatly as multiple members of the team (including myself) had problems expressing the difficulties we were having with the work at hand which slowed the project down to a halt at times. Being able to find these issues and working on them as a team from a earlier stage would have greatly increased the productivity in the group. Continuing on from this project I will make a conscious effort to increase productivity within the team by encouraging issues and team members concerns to be talked about in our regular weekly meet-ups. Hopefully doing this will help our work flow and increase my team working skill set in the next term. 

\section{Academic researching skills}

I've always felt that coming into university my biggest weakness lies not with programming itself but with academic writing at the university level as independent research has always been very hard work for me throughout higher education. A good example of this is the two pieces of academic assignments that we have been given to complete for this term in comp-150 and comp-110. The comp 150 essay about the subject of agile development in the game industry has been really difficult to create a question for. Although a small part of this was because although I have no issues with using agile I personally don't find it a compelling subject, it's mostly because of my lack of skill in using references correctly in the structure of my work and relating it to the question that I have posed. To increase my ability at creating academic essays I believe I need to organise a block of spare time in the week to dedicate to academic research and look back at how I have improved in the following term.
\section{Developing Programming knowledge}

Write about 200 words. As above.


\section{Conclusion}

Write your conclusion here. Though the conclusion should be brief, no more than 100 words, it should do more than merely summarise the report. Focus on the five SMART actions that you intend to take in order to overcome any challenges and/or obstacles. Contextualise how this will help you towards your intended career goal and how this may improve your project for the next semester.

\bibliographystyle{ieeetran}
\bibliography{references}

\end{document}
